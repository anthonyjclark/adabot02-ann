\section{Introduction}

% (1) Introductory paragraph: Very briefly: What is the problem and why is it relevant to the audience attending *THIS CONFERENCE*? Moreover, why is the problem hard, and what is your solution? You must be brief here. This forces you to boil down your contribution to its bare essence and communicate it directly.
Autonomous unmanned ground vehicles (UGV) provide an excellent solution to any task that requires searching or monitoring in environments that are too inconvenient or dangerous for humans.
%
Consider the \emph{search and rescue}  task: after a natural disaster a UGV can be used by first responders to help locate victims in unstable and hazardous locations.
%
Compared with an unmanned aerial vehicle (UAV), a UGV has a longer operating duration and can carry heavier sensors.
%
Ideally, a heterogeneous swarm of UAVs and UGV would coordinate on this task to gain the advantages of both search modes~\autocite{Kruijff.SearchAndRescue.ICFSR.2014}.
%
In this paper, however, we are only concerned with UGVs.


% (2) Background paragraph: Elaborate on why the problem is hard, critically examining prior work, trying to tease out one or two central shortcomings that your solution overcomes.
One issue that arises during the design of a UGV is how to ensure that the system can handle many different types of terrain.
%
Researches have invented several different methods for addressing the issue of mobility in varied terrain.
%
Specifically, robots have been designed with treaded wheels, tracks, legged-wheels (wheels are rimless, wheel spokes make contact with the ground), wheeled-legs (wheels are on the end of legs and the suspension is potentially actively actuated), or transformable wheels.
%
The device in this study, called the \emph{Adabot} (see \figref{fig:adabot}), includes transformable wheels that can smoothly be converted from a round wheel to a wheel, to a wheel with tire \emph{studs}, to a legged-wheel.


% (3) Transition paragraph: What keen insight did you apply to overcome the shortcomings of other approaches? Structure this paragraph like a syllogism: Whereas P and P => Q, infer Q.

% (4) Details paragraph: What technical challenges did you have to overcome and what kinds of validation did you perform?

% (5) Assessment paragraph: Assess your results and briefly state the broadly interesting conclusions that these results support. This may only take a couple of sentences. I usually then follow these sentences by an optional overview of the structure of the paper with interleaved section callouts.


\begin{figure}[!ht]
    \centering

    % \includegraphics[width=0.6\columnwidth]{figures/4-simulation/state-machine.png}
    \fbox{\crule[white]{\columnwidth}{2cm}}

    \caption{This is a temp caption.}
    \label{fig:adabot}
\end{figure}
