\section{Related Work}
\label{sec:related-work}

% \red{Robots are well suited to tasks that are too mundane, precise, or dangerous for humans.
% %
% Increasingly, autonomous robots are being developed to operate in rugged and dynamic environments.
% %
% Such terrain is particularly difficult for wheeled UGVs, which has prompted the design of many different suspensions and wheel configurations~\citep{Seeni.Mobility.InTech.2010}.
% %
% The challenge for these devices is to successfully navigate an area without prior knowledge of the obstacles and ground consistency.
% %
% Moreover, robotics engineers must choose values for all free parameters (\eg{}, the chassis and wheel dimensions).}


% Evolve UGVs (ER)
In the field of evolutionary robotics (ER), an evolutionary algorithm (EA) optimizes free variables of a given system~\citep{Silva.EvolComp.Issues-ER.2016}.
%
ER methods have been successfully applied to many different types of robotic systems (aerial, aquatic, walking, etc.).
%
% TODO change this citation
For example, we have previously used differential evolution to evolve adaptive neural networks and morphologies for a robotic fish \citep{Clark.GECCO.MFAC.2015,Clark.2015.BB.EvolutionaryMultiobjectiveDesign}, and \citet{Moore.2017.GECCO.Animat} evolved hierarchical controllers for segmented worm-like animats.
%
Although evolution has been regularly utilized at an abstract level to optimize wheeled-robot navigation processes (for example, see \citet{Gomes.2015.GECCO.Maze} and \citet{Lehman.2011.EC.AbandoningObjectivesEvolution}), it has not often been used to directly evolve UGV morphologies, and to the best of our knowledge this is the first study in which the characteristics of a transformable wheel are evolved.
%
% However, as identified by \citet{Bongard.2013.CACM.ER}, and in the work presented here, ER methods can be effectively applied to wheeled robots.


% Analyze neural networks
A large number of ER studies utilize ANNs to control mobile robots, including \emph{Evolving Virtual Creatures} \citep{Sims.1994.AL.Evolving3DMorphology}, which is considered one of the first ER works.
%
ANNs  provide several benefits when using an evolutionary method.
%
First, since ANNs are so-called universal approximators~\citep{Hornik.1989.NN.UniversalANN}, evolution often produces \emph{novel} and sometimes unintuitive results that may not have been found when creating a controller by hand~\citep{Bongard.2013.CACM.ER}.
%
And second, ANNs require a minimal amount of user design. Specifically, an evolutionary algorithm can automatically decide the importance of each input (sensor values) in the calculation of each output (actuation mechanisms)~\citep{Stanley.2002.EC.NEAT}.
%
The primary disadvantage of using an ANN is that it is considered a black-box system. That is, \emph{how} an ANN achieves its results is not often clear or analyzed.
%
Recently, however, some researchers have attempted to extract state machines from evolved neural networks.
%
For example, \citet{Wrobel.2017.SSCI.VSSNN} automatically generated a state machine with the same properties of an evolved spiking neural network.
